%**************************************%
%*    Generated from PreTeXt source   *%
%*    on 2020-06-21T21:57:23-04:00    *%
%*                                    *%
%*      https://pretextbook.org       *%
%*                                    *%
%**************************************%
\documentclass[oneside,10pt,]{book}
%% Custom Preamble Entries, early (use latex.preamble.early)
%% Default LaTeX packages
%%   1.  always employed (or nearly so) for some purpose, or
%%   2.  a stylewriter may assume their presence
\usepackage{geometry}
%% Some aspects of the preamble are conditional,
%% the LaTeX engine is one such determinant
\usepackage{ifthen}
%% etoolbox has a variety of modern conveniences
\usepackage{etoolbox}
\usepackage{ifxetex,ifluatex}
%% Raster graphics inclusion
\usepackage{graphicx}
%% Color support, xcolor package
%% Always loaded, for: add/delete text, author tools
%% Here, since tcolorbox loads tikz, and tikz loads xcolor
\PassOptionsToPackage{usenames,dvipsnames,svgnames,table}{xcolor}
\usepackage{xcolor}
%% begin: defined colors, via xcolor package, for styling
%% end: defined colors, via xcolor package, for styling
%% Colored boxes, and much more, though mostly styling
%% skins library provides "enhanced" skin, employing tikzpicture
%% boxes may be configured as "breakable" or "unbreakable"
%% "raster" controls grids of boxes, aka side-by-side
\usepackage{tcolorbox}
\tcbuselibrary{skins}
\tcbuselibrary{breakable}
\tcbuselibrary{raster}
%% We load some "stock" tcolorbox styles that we use a lot
%% Placement here is provisional, there will be some color work also
%% First, black on white, no border, transparent, but no assumption about titles
\tcbset{ bwminimalstyle/.style={size=minimal, boxrule=-0.3pt, frame empty,
colback=white, colbacktitle=white, coltitle=black, opacityfill=0.0} }
%% Second, bold title, run-in to text/paragraph/heading
%% Space afterwards will be controlled by environment,
%% independent of constructions of the tcb title
%% Places \blocktitlefont onto many block titles
\tcbset{ runintitlestyle/.style={fonttitle=\blocktitlefont\upshape\bfseries, attach title to upper} }
%% Spacing prior to each exercise, anywhere
\tcbset{ exercisespacingstyle/.style={before skip={1.5ex plus 0.5ex}} }
%% Spacing prior to each block
\tcbset{ blockspacingstyle/.style={before skip={2.0ex plus 0.5ex}} }
%% xparse allows the construction of more robust commands,
%% this is a necessity for isolating styling and behavior
%% The tcolorbox library of the same name loads the base library
\tcbuselibrary{xparse}
%% Hyperref should be here, but likes to be loaded late
%%
%% Inline math delimiters, \(, \), need to be robust
%% 2016-01-31:  latexrelease.sty  supersedes  fixltx2e.sty
%% If  latexrelease.sty  exists, bugfix is in kernel
%% If not, bugfix is in  fixltx2e.sty
%% See:  https://tug.org/TUGboat/tb36-3/tb114ltnews22.pdf
%% and read "Fewer fragile commands" in distribution's  latexchanges.pdf
\IfFileExists{latexrelease.sty}{}{\usepackage{fixltx2e}}
%% shorter subnumbers in some side-by-side require manipulations
\usepackage{xstring}
%% Text height identically 9 inches, text width varies on point size
%% See Bringhurst 2.1.1 on measure for recommendations
%% 75 characters per line (count spaces, punctuation) is target
%% which is the upper limit of Bringhurst's recommendations
\geometry{letterpaper,total={340pt,9.0in}}
%% Custom Page Layout Adjustments (use latex.geometry)
%% This LaTeX file may be compiled with pdflatex, xelatex, or lualatex executables
%% LuaTeX is not explicitly supported, but we do accept additions from knowledgeable users
%% The conditional below provides  pdflatex  specific configuration last
%% begin: engine-specific capabilities
\ifthenelse{\boolean{xetex} \or \boolean{luatex}}{%
%% begin: xelatex and lualatex-specific default configuration
\ifxetex\usepackage{xltxtra}\fi
%% realscripts is the only part of xltxtra relevant to lualatex 
\ifluatex\usepackage{realscripts}\fi
%% end:   xelatex and lualatex-specific default configuration
}{
%% begin: pdflatex-specific default configuration
%% We assume a PreTeXt XML source file may have Unicode characters
%% and so we ask LaTeX to parse a UTF-8 encoded file
%% This may work well for accented characters in Western language,
%% but not with Greek, Asian languages, etc.
%% When this is not good enough, switch to the  xelatex  engine
%% where Unicode is better supported (encouraged, even)
\usepackage[utf8]{inputenc}
%% end: pdflatex-specific default configuration
}
%% end:   engine-specific capabilities
%%
%% Fonts.  Conditional on LaTex engine employed.
%% Default Text Font: The Latin Modern fonts are
%% "enhanced versions of the [original TeX] Computer Modern fonts."
%% We use them as the default text font for PreTeXt output.
%% Default Monospace font: Inconsolata (aka zi4)
%% Sponsored by TUG: http://levien.com/type/myfonts/inconsolata.html
%% Loaded for documents with intentional objects requiring monospace
%% See package documentation for excellent instructions
%% fontspec will work universally if we use filename to locate OTF files
%% Loads the "upquote" package as needed, so we don't have to
%% Upright quotes might come from the  textcomp  package, which we also use
%% We employ the shapely \ell to match Google Font version
%% pdflatex: "varl" package option produces shapely \ell
%% pdflatex: "var0" package option produces plain zero (not used)
%% pdflatex: "varqu" package option produces best upright quotes
%% xelatex,lualatex: add OTF StylisticSet 1 for shapely \ell
%% xelatex,lualatex: add OTF StylisticSet 2 for plain zero (not used)
%% xelatex,lualatex: add OTF StylisticSet 3 for upright quotes
%%
%% Automatic Font Control
%% Portions of a document, are, or may, be affected by defined commands
%% These are perhaps more flexible when using  xelatex  rather than  pdflatex
%% The following definitions are meant to be re-defined in a style, using \renewcommand
%% They are scoped when employed (in a TeX group), and so should not be defined with an argument
\newcommand{\divisionfont}{\relax}
\newcommand{\blocktitlefont}{\relax}
\newcommand{\contentsfont}{\relax}
\newcommand{\pagefont}{\relax}
\newcommand{\tabularfont}{\relax}
\newcommand{\xreffont}{\relax}
\newcommand{\titlepagefont}{\relax}
%%
\ifthenelse{\boolean{xetex} \or \boolean{luatex}}{%
%% begin: font setup and configuration for use with xelatex
%% Generally, xelatex is necessary for non-Western fonts
%% fontspec package provides extensive control of system fonts,
%% meaning *.otf (OpenType), and apparently *.ttf (TrueType)
%% that live *outside* your TeX/MF tree, and are controlled by your *system*
%% (it is possible that a TeX distribution will place fonts in a system location)
%%
%% The fontspec package is the best vehicle for using different fonts in  xelatex
%% So we load it always, no matter what a publisher or style might want
%%
\usepackage{fontspec}
%%
%% begin: xelatex main font ("font-xelatex-main" template)
%% Latin Modern Roman is the default font for xelatex and so is loaded with a TU encoding
%% *in the format* so we can't touch it, only perhaps adjust it later
%% in one of two ways (then known by NFSS names such as "lmr")
%% (1) via NFSS with font family names such as "lmr" and "lmss"
%% (2) via fontspec with commands like \setmainfont{Latin Modern Roman}
%% The latter requires the font to be known at the system-level by its font name,
%% but will give access to OTF font features through optional arguments
%% https://tex.stackexchange.com/questions/470008/
%% where-and-how-does-fontspec-sty-specify-the-default-font-latin-modern-roman
%% http://tex.stackexchange.com/questions/115321
%% /how-to-optimize-latin-modern-font-with-xelatex
%%
%% end:   xelatex main font ("font-xelatex-main" template)
%% begin: xelatex mono font ("font-xelatex-mono" template)
%% (conditional on non-trivial uses being present in source)
\IfFontExistsTF{Inconsolatazi4-Regular.otf}{}{\GenericError{}{The font "Inconsolatazi4-Regular.otf" requested by PreTeXt output is not available.  Either a file cannot be located in default locations via a filename, or a font is not known by its name as part of your system.}{Consult the PreTeXt Guide for help with LaTeX fonts.}{}}
\IfFontExistsTF{Inconsolatazi4-Bold.otf}{}{\GenericError{}{The font "Inconsolatazi4-Bold.otf" requested by PreTeXt output is not available.  Either a file cannot be located in default locations via a filename, or a font is not known by its name as part of your system.}{Consult the PreTeXt Guide for help with LaTeX fonts.}{}}
\usepackage{zi4}
\setmonofont[BoldFont=Inconsolatazi4-Bold.otf,StylisticSet={1,3}]{Inconsolatazi4-Regular.otf}
%% end:   xelatex mono font ("font-xelatex-mono" template)
%% begin: xelatex font adjustments ("font-xelatex-style" template)
%% end:   xelatex font adjustments ("font-xelatex-style" template)
%%
%% Extensive support for other languages
\usepackage{polyglossia}
%% Set main/default language based on pretext/@xml:lang value
%% document language code is "en-US", US English
%% usmax variant has extra hypenation
\setmainlanguage[variant=usmax]{english}
%% Enable secondary languages based on discovery of @xml:lang values
%% Enable fonts/scripts based on discovery of @xml:lang values
%% Western languages should be ably covered by Latin Modern Roman
%% end:   font setup and configuration for use with xelatex
}{%
%% begin: font setup and configuration for use with pdflatex
%% begin: pdflatex main font ("font-pdflatex-main" template)
\usepackage{lmodern}
\usepackage[T1]{fontenc}
%% end:   pdflatex main font ("font-pdflatex-main" template)
%% begin: pdflatex mono font ("font-pdflatex-mono" template)
%% (conditional on non-trivial uses being present in source)
\usepackage[varqu,varl]{inconsolata}
%% end:   pdflatex mono font ("font-pdflatex-mono" template)
%% begin: pdflatex font adjustments ("font-pdflatex-style" template)
%% end:   pdflatex font adjustments ("font-pdflatex-style" template)
%% end:   font setup and configuration for use with pdflatex
}
%% Symbols, align environment, commutative diagrams, bracket-matrix
\usepackage{amsmath}
\usepackage{amscd}
\usepackage{amssymb}
%% allow page breaks within display mathematics anywhere
%% level 4 is maximally permissive
%% this is exactly the opposite of AMSmath package philosophy
%% there are per-display, and per-equation options to control this
%% split, aligned, gathered, and alignedat are not affected
\allowdisplaybreaks[4]
%% allow more columns to a matrix
%% can make this even bigger by overriding with  latex.preamble.late  processing option
\setcounter{MaxMatrixCols}{30}
%%
%%
%% Division Titles, and Page Headers/Footers
%% titlesec package, loading "titleps" package cooperatively
%% See code comments about the necessity and purpose of "explicit" option.
%% The "newparttoc" option causes a consistent entry for parts in the ToC 
%% file, but it is only effective if there is a \titleformat for \part.
%% "pagestyles" loads the  titleps  package cooperatively.
\usepackage[explicit, newparttoc, pagestyles]{titlesec}
%% The companion titletoc package for the ToC.
\usepackage{titletoc}
%% Fixes a bug with transition from chapters to appendices in a "book"
%% See generating XSL code for more details about necessity
\newtitlemark{\chaptertitlename}
%% begin: customizations of page styles via the modal "titleps-style" template
%% Designed to use commands from the LaTeX "titleps" package
%% Plain pages should have the same font for page numbers
\renewpagestyle{plain}{%
\setfoot{}{\pagefont\thepage}{}%
}%
%% Single pages as in default LaTeX
\renewpagestyle{headings}{%
\sethead{\pagefont\slshape\MakeUppercase{\ifthechapter{\chaptertitlename\space\thechapter.\space}{}\chaptertitle}}{}{\pagefont\thepage}%
}%
\pagestyle{headings}
%% end: customizations of page styles via the modal "titleps-style" template
%%
%% Create globally-available macros to be provided for style writers
%% These are redefined for each occurence of each division
\newcommand{\divisionnameptx}{\relax}%
\newcommand{\titleptx}{\relax}%
\newcommand{\subtitleptx}{\relax}%
\newcommand{\shortitleptx}{\relax}%
\newcommand{\authorsptx}{\relax}%
\newcommand{\epigraphptx}{\relax}%
%% Create environments for possible occurences of each division
%% Environment for a PTX "preface" at the level of a LaTeX "chapter"
\NewDocumentEnvironment{preface}{mmmmmm}
{%
\renewcommand{\divisionnameptx}{Preface}%
\renewcommand{\titleptx}{#1}%
\renewcommand{\subtitleptx}{#2}%
\renewcommand{\shortitleptx}{#3}%
\renewcommand{\authorsptx}{#4}%
\renewcommand{\epigraphptx}{#5}%
\chapter*{#1}%
\addcontentsline{toc}{chapter}{#3}
\label{#6}%
}{}%
%% Environment for a PTX "chapter" at the level of a LaTeX "chapter"
\NewDocumentEnvironment{chapterptx}{mmmmmm}
{%
\renewcommand{\divisionnameptx}{Chapter}%
\renewcommand{\titleptx}{#1}%
\renewcommand{\subtitleptx}{#2}%
\renewcommand{\shortitleptx}{#3}%
\renewcommand{\authorsptx}{#4}%
\renewcommand{\epigraphptx}{#5}%
\chapter[{#3}]{#1}%
\label{#6}%
}{}%
%% Environment for a PTX "section" at the level of a LaTeX "section"
\NewDocumentEnvironment{sectionptx}{mmmmmm}
{%
\renewcommand{\divisionnameptx}{Section}%
\renewcommand{\titleptx}{#1}%
\renewcommand{\subtitleptx}{#2}%
\renewcommand{\shortitleptx}{#3}%
\renewcommand{\authorsptx}{#4}%
\renewcommand{\epigraphptx}{#5}%
\section[{#3}]{#1}%
\label{#6}%
}{}%
%%
%% Styles for six traditional LaTeX divisions
\titleformat{\part}[display]
{\divisionfont\Huge\bfseries\centering}{\divisionnameptx\space\thepart}{30pt}{\Huge#1}
[{\Large\centering\authorsptx}]
\titleformat{\chapter}[display]
{\divisionfont\huge\bfseries}{\divisionnameptx\space\thechapter}{20pt}{\Huge#1}
[{\Large\authorsptx}]
\titleformat{name=\chapter,numberless}[display]
{\divisionfont\huge\bfseries}{}{0pt}{#1}
[{\Large\authorsptx}]
\titlespacing*{\chapter}{0pt}{50pt}{40pt}
\titleformat{\section}[hang]
{\divisionfont\Large\bfseries}{\thesection}{1ex}{#1}
[{\large\authorsptx}]
\titleformat{name=\section,numberless}[block]
{\divisionfont\Large\bfseries}{}{0pt}{#1}
[{\large\authorsptx}]
\titlespacing*{\section}{0pt}{3.5ex plus 1ex minus .2ex}{2.3ex plus .2ex}
\titleformat{\subsection}[hang]
{\divisionfont\large\bfseries}{\thesubsection}{1ex}{#1}
[{\normalsize\authorsptx}]
\titleformat{name=\subsection,numberless}[block]
{\divisionfont\large\bfseries}{}{0pt}{#1}
[{\normalsize\authorsptx}]
\titlespacing*{\subsection}{0pt}{3.25ex plus 1ex minus .2ex}{1.5ex plus .2ex}
\titleformat{\subsubsection}[hang]
{\divisionfont\normalsize\bfseries}{\thesubsubsection}{1em}{#1}
[{\small\authorsptx}]
\titleformat{name=\subsubsection,numberless}[block]
{\divisionfont\normalsize\bfseries}{}{0pt}{#1}
[{\normalsize\authorsptx}]
\titlespacing*{\subsubsection}{0pt}{3.25ex plus 1ex minus .2ex}{1.5ex plus .2ex}
\titleformat{\paragraph}[hang]
{\divisionfont\normalsize\bfseries}{\theparagraph}{1em}{#1}
[{\small\authorsptx}]
\titleformat{name=\paragraph,numberless}[block]
{\divisionfont\normalsize\bfseries}{}{0pt}{#1}
[{\normalsize\authorsptx}]
\titlespacing*{\paragraph}{0pt}{3.25ex plus 1ex minus .2ex}{1.5em}
%%
%% Styles for five traditional LaTeX divisions
\titlecontents{part}%
[0pt]{\contentsmargin{0em}\addvspace{1pc}\contentsfont\bfseries}%
{\Large\thecontentslabel\enspace}{\Large}%
{}%
[\addvspace{.5pc}]%
\titlecontents{chapter}%
[0pt]{\contentsmargin{0em}\addvspace{1pc}\contentsfont\bfseries}%
{\large\thecontentslabel\enspace}{\large}%
{\hfill\bfseries\thecontentspage}%
[\addvspace{.5pc}]%
\dottedcontents{section}[3.8em]{\contentsfont}{2.3em}{1pc}%
\dottedcontents{subsection}[6.1em]{\contentsfont}{3.2em}{1pc}%
\dottedcontents{subsubsection}[9.3em]{\contentsfont}{4.3em}{1pc}%
%%
%% Begin: Semantic Macros
%% To preserve meaning in a LaTeX file
%%
%% \mono macro for content of "c", "cd", "tag", etc elements
%% Also used automatically in other constructions
%% Simply an alias for \texttt
%% Always defined, even if there is no need, or if a specific tt font is not loaded
\newcommand{\mono}[1]{\texttt{#1}}
%%
%% Following semantic macros are only defined here if their
%% use is required only in this specific document
%%
%% Used for warnings, typically bold and italic
\newcommand{\alert}[1]{\textbf{\textit{#1}}}
%% Used for inline definitions of terms
\newcommand{\terminology}[1]{\textbf{#1}}
%% End: Semantic Macros
%% Division Numbering: Chapters, Sections, Subsections, etc
%% Division numbers may be turned off at some level ("depth")
%% A section *always* has depth 1, contrary to us counting from the document root
%% The latex default is 3.  If a larger number is present here, then
%% removing this command may make some cross-references ambiguous
%% The precursor variable $numbering-maxlevel is checked for consistency in the common XSL file
\setcounter{secnumdepth}{3}
%%
%% AMS "proof" environment is no longer used, but we leave previously
%% implemented \qedhere in place, should the LaTeX be recycled
\newcommand{\qedhere}{\relax}
%%
%% A faux tcolorbox whose only purpose is to provide common numbering
%% facilities for most blocks (possibly not projects, 2D displays)
%% Controlled by  numbering.theorems.level  processing parameter
\newtcolorbox[auto counter, number within=section]{block}{}
%%
%% This document is set to number PROJECT-LIKE on a separate numbering scheme
%% So, a faux tcolorbox whose only purpose is to provide this numbering
%% Controlled by  numbering.projects.level  processing parameter
\newtcolorbox[auto counter, number within=section]{project-distinct}{}
%% A faux tcolorbox whose only purpose is to provide common numbering
%% facilities for 2D displays which are subnumbered as part of a "sidebyside"
\newtcolorbox[auto counter, number within=tcb@cnt@block, number freestyle={\noexpand\thetcb@cnt@block(\noexpand\alph{\tcbcounter})}]{subdisplay}{}
%%
%% tcolorbox, with styles, for THEOREM-LIKE
%%
%% theorem: fairly simple numbered block/structure
\tcbset{ theoremstyle/.style={bwminimalstyle, runintitlestyle, blockspacingstyle, after title={\space}, } }
\newtcolorbox[use counter from=block]{theorem}[3]{title={{Theorem~\thetcbcounter\notblank{#1#2}{\space}{}\notblank{#1}{\space#1}{}\notblank{#2}{\space(#2)}{}}}, phantomlabel={#3}, breakable, parbox=false, after={\par}, fontupper=\itshape, theoremstyle, }
%% lemma: fairly simple numbered block/structure
\tcbset{ lemmastyle/.style={bwminimalstyle, runintitlestyle, blockspacingstyle, after title={\space}, } }
\newtcolorbox[use counter from=block]{lemma}[3]{title={{Lemma~\thetcbcounter\notblank{#1#2}{\space}{}\notblank{#1}{\space#1}{}\notblank{#2}{\space(#2)}{}}}, phantomlabel={#3}, breakable, parbox=false, after={\par}, fontupper=\itshape, lemmastyle, }
%%
%% tcolorbox, with styles, for DEFINITION-LIKE
%%
%% definition: fairly simple numbered block/structure
\tcbset{ definitionstyle/.style={bwminimalstyle, runintitlestyle, blockspacingstyle, after title={\space}, after upper={\space\space\hspace*{\stretch{1}}\(\lozenge\)}, } }
\newtcolorbox[use counter from=block]{definition}[2]{title={{Definition~\thetcbcounter\notblank{#1}{\space\space#1}{}}}, phantomlabel={#2}, breakable, parbox=false, after={\par}, definitionstyle, }
%%
%% tcolorbox, with styles, for REMARK-LIKE
%%
%% remark: fairly simple numbered block/structure
\tcbset{ remarkstyle/.style={bwminimalstyle, runintitlestyle, blockspacingstyle, after title={\space}, } }
\newtcolorbox[use counter from=block]{remark}[2]{title={{Remark~\thetcbcounter\notblank{#1}{\space\space#1}{}}}, phantomlabel={#2}, breakable, parbox=false, after={\par}, remarkstyle, }
%%
%% tcolorbox, with styles, for inline exercises
%%
%% inlineexercise: fairly simple numbered block/structure
\tcbset{ inlineexercisestyle/.style={bwminimalstyle, runintitlestyle, blockspacingstyle, after title={\space}, } }
\newtcolorbox[use counter from=block]{inlineexercise}[2]{title={{Checkpoint~\thetcbcounter\notblank{#1}{\space\space#1}{}}}, phantomlabel={#2}, breakable, parbox=false, after={\par}, inlineexercisestyle, }
%%
%% tcolorbox, with styles, for PROJECT-LIKE
%%
%% activity: fairly simple numbered block/structure
\tcbset{ activitystyle/.style={bwminimalstyle, runintitlestyle, blockspacingstyle, after title={\space}, } }
\newtcolorbox[use counter from=project-distinct]{activity}[2]{title={{Activity~\thetcbcounter\notblank{#1}{\space\space#1}{}}}, phantomlabel={#2}, breakable, parbox=false, after={\par}, activitystyle, }
%%
%% tcolorbox, with styles, for GOAL-LIKE
%%
%% objectives: early in a subdivision, introduction/list/conclusion
\tcbset{ objectivesstyle/.style={bwminimalstyle, blockspacingstyle, fonttitle=\blocktitlefont\large\bfseries, toprule=0.1ex, toptitle=0.5ex, top=2ex, bottom=0.5ex, bottomrule=0.1ex} }
\newtcolorbox{objectives}[2]{title={#1}, phantomlabel={#2}, breakable, parbox=false, objectivesstyle}
%%
%% tcolorbox, with styles, for FIGURE-LIKE
%%
%% figureptx: 2-D display structure
\tcbset{ figureptxstyle/.style={bwminimalstyle, middle=1ex, blockspacingstyle, fontlower=\blocktitlefont} }
\newtcolorbox[use counter from=block]{figureptx}[3]{lower separated=false, before lower={{\textbf{Figure~\thetcbcounter}\space#1}}, phantomlabel={#2}, unbreakable, parbox=false, figureptxstyle, }
%%
%% xparse environments for introductions and conclusions of divisions
%%
%% introduction: in a structured division
\NewDocumentEnvironment{introduction}{m}
{\notblank{#1}{\noindent\textbf{#1}\space}{}}{\par\medskip}
%% Localize LaTeX supplied names (possibly none)
\renewcommand*{\chaptername}{Chapter}
%% Equation Numbering
%% Controlled by  numbering.equations.level  processing parameter
%% No adjustment here implies document-wide numbering
\numberwithin{equation}{section}
%% "tcolorbox" environment for a single image, occupying entire \linewidth
%% arguments are left-margin, width, right-margin, as multiples of
%% \linewidth, and are guaranteed to be positive and sum to 1.0
\tcbset{ imagestyle/.style={bwminimalstyle} }
\NewTColorBox{image}{mmm}{imagestyle,left skip=#1\linewidth,width=#2\linewidth}
%% Multiple column, column-major lists
\usepackage{multicol}
%% More flexible list management, esp. for references
%% But also for specifying labels (i.e. custom order) on nested lists
\usepackage{enumitem}
%% hyperref driver does not need to be specified, it will be detected
%% Footnote marks in tcolorbox have broken linking under
%% hyperref, so it is necessary to turn off all linking
%% It *must* be given as a package option, not with \hypersetup
\usepackage[hyperfootnotes=false]{hyperref}
%% Hyperlinking active in electronic PDFs, all links solid and blue
\hypersetup{colorlinks=true,linkcolor=blue,citecolor=blue,filecolor=blue,urlcolor=blue}
\hypersetup{pdftitle={Introduction to Discrete Mathematics}}
%% If you manually remove hyperref, leave in this next command
\providecommand\phantomsection{}
%% Graphics Preamble Entries
\usepackage{tikz}
\usepackage{pgfplots}
\pgfplotsset{compat=1.15}
\usetikzlibrary{arrows,arrows.meta,decorations.markings}
\usetikzlibrary{graphs,graphs.standard}
\usepackage{makecell}
\usepackage{setspace}
\usepackage{mdframed}
\usepackage{venndiagram}
\DeclareSymbolFont{extraup}{U}{zavm}{m}{n}
\DeclareMathSymbol{\varheart}{\mathalpha}{extraup}{86}
\DeclareMathSymbol{\vardiamond}{\mathalpha}{extraup}{87}

\newcommand{\vheart}{\textcolor{red}{\varheart}}
\newcommand{\vdia}{\textcolor{red}{\vardiamond}}
\newcommand{\Mod}[1]{\ \left(\mathrm{mod}\ #1\right)}
\newcommand{\mmod}[1]{\ \mathrm{\bf mod}\ #1}
%% If tikz has been loaded, replace ampersand with \amp macro
\ifdefined\tikzset
    \tikzset{ampersand replacement = \amp}
\fi
%% tcolorbox styles for sidebyside layout
\tcbset{ sbsstyle/.style={raster before skip=2.0ex, raster equal height=rows, raster force size=false} }
\tcbset{ sbspanelstyle/.style={bwminimalstyle, fonttitle=\blocktitlefont} }
%% Enviroments for side-by-side and components
%% Necessary to use \NewTColorBox for boxes of the panels
%% "newfloat" environment to squash page-breaks within a single sidebyside
%% "xparse" environment for entire sidebyside
\NewDocumentEnvironment{sidebyside}{mmmm}
  {\begin{tcbraster}
    [sbsstyle,raster columns=#1,
    raster left skip=#2\linewidth,raster right skip=#3\linewidth,raster column skip=#4\linewidth]}
  {\end{tcbraster}}
%% "tcolorbox" environment for a panel of sidebyside
\NewTColorBox{sbspanel}{mO{top}}{sbspanelstyle,width=#1\linewidth,valign=#2}
%% extpfeil package for certain extensible arrows,
%% as also provided by MathJax extension of the same name
%% NB: this package loads mtools, which loads calc, which redefines
%%     \setlength, so it can be removed if it seems to be in the 
%%     way and your math does not use:
%%     
%%     \xtwoheadrightarrow, \xtwoheadleftarrow, \xmapsto, \xlongequal, \xtofrom
%%     
%%     we have had to be extra careful with variable thickness
%%     lines in tables, and so also load this package late
\usepackage{extpfeil}
%% Custom Preamble Entries, late (use latex.preamble.late)
%% Begin: Author-provided packages
%% (From  docinfo/latex-preamble/package  elements)
%% End: Author-provided packages
%% Begin: Author-provided macros
%% (From  docinfo/macros  element)
%% Plus three from MBX for XML characters

\newcommand{\lt}{<}
\newcommand{\gt}{>}
\newcommand{\amp}{&}
%% End: Author-provided macros
\begin{document}
\frontmatter
%% begin: half-title
\thispagestyle{empty}
{\titlepagefont\centering
\vspace*{0.28\textheight}
{\Huge Introduction to Discrete Mathematics}\\[2\baselineskip]
{\LARGE MAT202 Course Notes}\\
}
\clearpage
%% end:   half-title
%% begin: adcard
\thispagestyle{empty}
\null%
\clearpage
%% end:   adcard
%% begin: title page
%% Inspired by Peter Wilson's "titleDB" in "titlepages" CTAN package
\thispagestyle{empty}
{\titlepagefont\centering
\vspace*{0.14\textheight}
%% Target for xref to top-level element is ToC
\addtocontents{toc}{\protect\hypertarget{x:book:MAT202notes}{}}
{\Huge Introduction to Discrete Mathematics}\\[\baselineskip]
{\LARGE MAT202 Course Notes}\\[3\baselineskip]
{\Large TJ Yusun}\\[0.5\baselineskip]
{\Large University of Toronto Mississauga\\
Mississauga, ON, Canada}\\[3\baselineskip]
{\Large June 21, 2020}\\}
\clearpage
%% end:   title page
%% begin: copyright-page
\thispagestyle{empty}
\hypertarget{x:colophon:colophon}{}\vspace*{\stretch{2}}
\noindent{\bfseries Edition}: Fall\slash{}Winter 2020-2021\par\medskip
\noindent{\bfseries Website}: \href{https:\slash{}\slash{}tjyusun.com\slash{}itdm\slash{}MAT202notes}{\mono{tjyusun.com/itdm}}\par\medskip
\noindent\textcopyright{}2020\quad{}Timothy Yusun\\[0.5\baselineskip]
Permission is granted to copy, distribute, and\slash{}or modify this document under the terms of the etc etc etc.\par\medskip
\vspace*{\stretch{1}}
\null\clearpage
%% end:   copyright-page
%
%
\typeout{************************************************}
\typeout{Preface  Preface}
\typeout{************************************************}
%
\begin{preface}{Preface}{}{Preface}{}{}{g:preface:id254279}
The word \emph{discrete} in the title of our course means \emph{separate}; something that is \emph{not smooth}. In the study of discrete mathematics we will typically concern ourselves with discrete objects such as the integers, graphs, finite and countable sets. (In contrast, excluded from this are objects that may vary continuously, such as those ones covered in trigonometry, calculus, and Euclidean geometry.)%
\end{preface}
%% begin: table of contents
%% Adjust Table of Contents
\setcounter{tocdepth}{1}
\renewcommand*\contentsname{Contents}
\tableofcontents
%% end:   table of contents
\mainmatter
%
%
\typeout{************************************************}
\typeout{Chapter 1 Review of MAT102}
\typeout{************************************************}
%
\begin{chapterptx}{Review of MAT102}{}{Review of MAT102}{}{}{x:chapter:chap-review-mat102}
\begin{introduction}{}%
Many of the concepts you learned in your MAT102 course will be useful in MAT202; in this chapter we briefly review some definitions and results, and present some exercises to warm up for the rest of the course! Material in this chapter is based on \emph{MAT102H5 Introduction to Mathematical Proofs} by \emph{Shay Fuchs.}%
\end{introduction}%
%
%
\typeout{************************************************}
\typeout{Section 1.1 Sets and Functions}
\typeout{************************************************}
%
\begin{sectionptx}{Sets and Functions}{}{Sets and Functions}{}{}{x:section:sec-sets-and-functions}
A \terminology{set} is is just a collection of objects (where the order in which the objects are listed does not matter). The following set operations should be familiar to you: intersection \(A \cap B\), union \(A \cup B\), complement \(A^c\), difference \(A \ \backslash \ B\), and Cartesian product \(A \times B\).%
\par
We also recall that proving the set \(A\) is a subset of the set \(B\) simply necessitates showing that any element of \(A\) can also be found in \(B\).%
\begin{definition}{Set Inclusion and Equality.}{x:definition:def-set-inclusion-equality}%
If \(A\) and \(B\) are sets in some universe \(U\), then we say \(A\) \terminology{is a subset of} \(B\), denoted by \(A \subseteq B\), if%
\begin{equation*}
(\forall x \in U)(x \in A \Rightarrow x \in B)\text{.}
\end{equation*}
We say that \(A\) and \(B\) are \terminology{equal} as sets if \(A \subseteq B\) and \(B \subseteq A\) both hold. This means that%
\begin{equation*}
(\forall x \in U)(x \in A \Leftrightarrow x \in B)\text{.}
\end{equation*}
%
\end{definition}
\begin{inlineexercise}{}{x:exercise:exe-review-set-inclusion}%
Define%
\begin{equation*}
A = \{k \in \mathbb{Z} : k = 6s + 3 \text{ for some } s \in \mathbb{Z}\}
\end{equation*}
and%
\begin{equation*}
B = \{m \in \mathbb{Z} : m = 3t \text{ for some } t \in \mathbb{Z}\}\text{.}
\end{equation*}
Prove that \(A \subseteq B\) holds.%
\par\smallskip%
\noindent\textbf{\blocktitlefont Hint}.\hypertarget{g:hint:id254626}{}\quad{}Pick an arbitrary element in \(A\), call it \(x\). Then you know \(x = 6s + 3\) for some integer \(s\). Can you express \(x\) in the form \(3t\) where \(t\) is an integer?%
\end{inlineexercise}
We will use the standard notation for these sets of numbers:%
\begin{align*}
\mathbb{N} \amp = \{1,2,3,\ldots,\}\\
\mathbb{Z} \amp = \{\ldots,-2,-1,0,1,2,\ldots\}\\
\mathbb{Q} \amp = \left\{\dfrac{p}{q} : p,q \in \mathbb{Z}\right\}\\
\mathbb{R} \amp = (-\infty,\infty), \text{ the set of real numbers.}
\end{align*}
Intervals of real numbers are denoted by \((a,b), [a,b],\) and other combinations, with \(-\infty\) or \(\infty\) as one of both of the endpoints.%
\begin{remark}{}{g:remark:id241453}%
Interval notation is used to refer to sets of real numbers. It is \emph{incorrect}, for instance, to say that \((-2,4) = \{-1,0,1,2,3\}\), or that \(\{0,1,2,3,\ldots\} = [0,\infty)\). Watch your notation!%
\end{remark}
\begin{definition}{Function.}{x:definition:def-function}%
A \terminology{function}%
\begin{equation*}
f: A \rightarrow B
\end{equation*}
is a rule that takes elements from its \terminology{domain} \(A\) and assigns to each one an element from the \terminology{codomain} \(B\).%
\end{definition}
\begin{definition}{Injective, surjective, bijective.}{x:definition:def-injective}%
A function \(f: A \rightarrow B\) is%
\begin{itemize}[label=\textbullet]
\item{}\terminology{injective} if for every \(x_1 \ne x_2 \in A\), \(f(x_1) \ne f(x_2)\).%
\item{}\terminology{surjective} if for every \(y \in B\), there exists an \(x \in A\) so that \(f(x) = y\).%
\item{}\terminology{bijective} if for every \(x_1 \ne x_2 \in A\), \(f(x_1) \ne f(x_2)\).%
\end{itemize}
%
\end{definition}
\begin{inlineexercise}{}{g:exercise:id241762}%
For each function, determine if it is injective, surjective, bijective, or none of these:%
%
\begin{enumerate}[label=(\alph*)]
\item{}\(f: \mathbb{R} \rightarrow (0,+\infty)\), \(f(x) = \sqrt{x^2+1}\)%
\item{}\(g: \mathbb{N} \rightarrow \mathbb{N}\), \(g(m) = 3m + 7m^2\)%
\item{}\(h: \mathbb{Z} \times \mathbb{Z} \rightarrow \mathbb{Z}\), \(h(a,b) = \dfrac{ab(b-1)}{2}\)%
\end{enumerate}
\par\smallskip%
\noindent\textbf{\blocktitlefont Answer}.\hypertarget{g:answer:id241797}{}\quad{}1. none, 2. injective, 3. surjective.%
\end{inlineexercise}
\begin{definition}{Composition.}{x:definition:def-composition}%
Given two functions \(f: A \rightarrow B\) and \(g: B \rightarrow C\), the \terminology{composition} \(g \circ f: A \rightarrow C\) is defined as the function%
\begin{equation*}
(g \circ f)(x) = g(f(x))
\end{equation*}
for all \(x \in A\).%
\end{definition}
\begin{theorem}{}{}{x:theorem:thm-composition-of-bijections}%
The composition of two (injections, surjections, bijections) is a(n) (injection, surjection, bijection).%
\end{theorem}
\begin{inlineexercise}{}{g:exercise:id241881}%
Prove \hyperref[x:theorem:thm-composition-of-bijections]{Theorem~{\xreffont\ref{x:theorem:thm-composition-of-bijections}}}.%
\end{inlineexercise}
Later in the course we will learn techniques for counting objects and proving that two sets have the same number of elements; the notion of cardinality will be a useful tool to remember.%
\begin{definition}{}{x:definition:def-cardinality}%
Two sets \(A\) and \(B\) are said to have the \terminology{same cardinality}, written as%
\begin{equation*}
|A| = |B|\text{,}
\end{equation*}
if there exists a bijection between them.%
\par
We also say \(A\) has cardinality \terminology{less than or equal to} the cardinality of \(B\), written as%
\begin{equation*}
|A| \leq |B|\text{,}
\end{equation*}
if there exists an injective function from \(A\) to \(B\).%
\end{definition}
\begin{inlineexercise}{}{g:exercise:id241987}%
Prove that the set of odd integers%
\begin{equation*}
O = \{\ldots,-3,-1,1,3,\ldots\}
\end{equation*}
has the same cardinality as \(\mathbb{N}\).%
\par\smallskip%
\noindent\textbf{\blocktitlefont Hint}.\hypertarget{g:hint:id242035}{}\quad{}Construct a bijection from \(O\) to \(\mathbb{N}\).%
\end{inlineexercise}
\begin{definition}{Power Set.}{x:definition:def-power-set}%
Let \(A\) be a set. The \terminology{power set} of \(A\), denoted by \(P(A)\), is the set%
\begin{equation*}
P(A) = \{X : X \subseteq A\}\text{,}
\end{equation*}
that is, it contains all subsets of \(A\).%
\end{definition}
\begin{inlineexercise}{}{g:exercise:id242097}%
Prove that if \(A\) is finite, then%
\begin{equation*}
|P(A)| = 2^{|A|}\text{.}
\end{equation*}
%
\end{inlineexercise}
\end{sectionptx}
%
%
\typeout{************************************************}
\typeout{Section 1.2 Logic and Proof Techniques}
\typeout{************************************************}
%
\begin{sectionptx}{Logic and Proof Techniques}{}{Logic and Proof Techniques}{}{}{x:section:sec-logic-and-proof-techniques}
Mathematical statements can typically be phrased as an implication \(P \Rightarrow Q\),  read as \emph{if \(P\), then \(Q\)}, where \(P\) or \(Q\) may be complex statements themselves that involve conjunctions (and), disjunctions (or), negations, quantifiers, even implications. There are various ways in which an implication can be proven true, and there is no hard and fast rule that dictates which proof method to use given a particular problem. In MAT102 you were introduced to the following proof techniques:%
\par
%
\begin{itemize}[label=\textbullet]
\item{}Direct proof: Assume \(P\) is true, then prove \(Q\) is true.%
\item{}Contrapositive: Assume \(\neg Q\) is true, then prove \(\neg P\) is true.%
\item{}Contradiction: Assume the conclusion is false, then use this to arrive at a statement that contradicts one of the assumptions.%
\end{itemize}
%
\begin{activity}{}{x:activity:exe-review-proofs}%
Prove each statement, noting which proof technique you used. Explain all your steps clearly, as if you are writing for the current batch of MAT102 students.%
\begin{enumerate}[label=(\alph*)]
\item{}The sum of two odd numbers is even.%
\item{}The square of an even number is divisible by 4.%
\item{}The equation \(x^3 + x + 1 = 0\) has no rational solutions.%
\item{}For integer \(n\), if \(n^3 + 5\) is odd, then \(n\) is even.%
\item{}There is no smallest positive rational number.%
\item{}Every multiple of 4 can be written as \(1 + (-1)^n(2n-1)\) for some \(n \in \mathbb{N}\).%
\item{}The sum of a rational number and an irrational number is irrational.%
\item{}A three-digit natural number is divisible by 9 if and only if the sum of its digits is divisible by 9.%
\item{}If \(A\) and \(B\) are defined as in \hyperref[x:exercise:exe-review-set-inclusion]{Checkpoint~{\xreffont\ref{x:exercise:exe-review-set-inclusion}}}, then \(B \not\subseteq A\).%
\end{enumerate}
%
\end{activity}
Many of these statements are \emph{quantified} universally, which means it involves some variable (say \(n\)), and you need to prove the claim holds for all relevant values of the variable (say \(n \in \mathbb{N}\)). For 1, 2, and 4, for example, the relevant quantities are integers; the statements need to be proven for all integers.%
\par
We can use mathematical induction to prove that a statement is true for all natural numbers.%
\begin{theorem}{Principle of Mathematical Induction.}{}{x:theorem:thm-induction}%
Let \(P(n)\) be a predicate defined for \(n \in \mathbb{N}\). If the following conditions hold:%
\begin{enumerate}[label=(\alph*)]
\item{}\(P(1)\) is true;%
\item{}For all \(k \in \mathbb{N}\), \(P(k) \Rightarrow P(k+1)\) is true.%
\end{enumerate}
then \(P(n)\) is true for all \(n \in \mathbb{N}\).%
\par
One can also replace the second condition with the following:%
\par
b.\textasteriskcentered{} For all \(k \in \mathbb{N}\), \([P(1) \wedge P(2) \wedge \cdots \wedge P(k)] \Rightarrow P(k+1)\) is true.%
\par
This is called \terminology{strong induction}, where one assumes the induction step holds for all natural numbers from 1 to \(k\) in order to prove the claim for \(k+1\).%
\end{theorem}
Depending on what is being proved, one may need to make slight modifications to the standard technique: e.g. changing\slash{}adding to the base case, or ``skipping'' from \(k\) to \(k+2\) in the case when one only has to prove the claim for every other natural number starting from the base case.%
\begin{inlineexercise}{}{x:exercise:exe-review-induction}%
Prove the following statements using induction:%
%
\begin{enumerate}[label=(\alph*)]
\item{}\(1^2 + 2^2 + 3^2 + \cdots + n^2 = \frac{1}{6}n(n+1)(2n+1)\) for all \(n \in \mathbb{N}\).%
\item{}\(2^n \geq n^2\) for all \(n \in \mathbb{N}, n \geq 4\).%
\item{}\(4^{2n} -1\) is divisible by 5 for every \(n \in \mathbb{N}\).%
\end{enumerate}
\end{inlineexercise}
\begin{inlineexercise}{}{x:exercise:exe-review-induction-fibonacci}%
The \terminology{Fibonacci sequence} \(\{F_n\}\) is defined recursively as%
\begin{equation*}
\begin{cases} F_n = F_{n-1} + F_{n-2}, n \geq 3 \\
F_1 = F_2 = 1 \end{cases}\text{.}
\end{equation*}
Prove that%
\begin{equation*}
F_n = \frac{1}{\sqrt{5}}\left(\frac{1+\sqrt{5}}{2}\right)^n - \frac{1}{\sqrt{5}}\left(\frac{1-\sqrt{5}}{2}\right)^n
\end{equation*}
using strong induction.%
\end{inlineexercise}
\begin{inlineexercise}{}{x:exercise:exe-review-induction-tiling}%
Let \(T_n\) be the number of ways one can tile a \(2 \times n\) grid with \(1 \times 2\) rectangles. For example, \(T_2 = 2\) since there are two tilings of a \(2 \times 2\) grid using only \(1 \times 2\) rectangles.%
\begin{figureptx}{The two tilings of a two-by-two grid.}{x:figure:exe-tiling-fig}{}%
\begin{image}{0.25}{0.5}{0.25}%
\resizebox{\linewidth}{!}{%
\begin{tikzpicture}
\draw[-,color=black,dashed] (0,0) grid (2,2);
\draw[-,thick] (0.05,0.05)--(1,0.05)--(1.95,0.05)--(1.95,0.95)--(1,0.95)--(0.05,0.95)--(0.05,0.05);
\draw[-,thick] (0.05,1.05)--(1,1.05)--(1.95,1.05)--(1.95,1.95)--(1,1.95)--(0.05,1.95)--(0.05,1.05);
\draw[-,color=black,dashed] (3,0) grid (5,2);
\draw[-,thick] (3.05,0.05)--(3.05,1)--(3.05,1.95)--(3.95,1.95)--(3.95,1)--(3.95,0.05)--(3.05,0.05);
\draw[-,thick] (4.05,0.05)--(4.05,1)--(4.05,1.95)--(4.95,1.95)--(4.95,1)--(4.95,0.05)--(4.05,0.05);
\end{tikzpicture}
}%
\end{image}%
\tcblower
\end{figureptx}%
Find a recurrence relation for \(T_n\) and prove that \(T_n = F_{n+1}\) as defined in \hyperref[x:exercise:exe-review-induction-fibonacci]{Checkpoint~{\xreffont\ref{x:exercise:exe-review-induction-fibonacci}}}.%
\end{inlineexercise}
\end{sectionptx}
%
%
\typeout{************************************************}
\typeout{Section 1.3 Integers and Divisibility}
\typeout{************************************************}
%
\begin{sectionptx}{Integers and Divisibility}{}{Integers and Divisibility}{}{}{x:section:sec-integers-divisibility}
For completeness we restate here the definition of divisibility and the Division Algorithm.%
\begin{definition}{}{x:definition:def-divisibility}%
Let \(a \in \mathbb{Z}\) and \(b \in \mathbb{Z} \setminus \{0\}\). We say that \(a\) \terminology{is divisible by} \(b\), or \(b\) \terminology{divides} \(a\), denoted by%
\begin{equation*}
b \mid a\text{,}
\end{equation*}
if there exists \(m \in \mathbb{Z}\) such that \(a = mb\).%
\par
If \(b\) is not divisible by \(a\), then we write \(b \nmid a\).%
\par
We say that the natural number \(p\) is a \terminology{prime number} if the only natural numbers that divide \(p\) are \(1\) and \(p\).%
\end{definition}
\begin{theorem}{}{}{x:theorem:thm-division-algorithm}%
Let \(a,b \in \mathbb{N}\). Then there exist unique \(q\) and \(r\) that satisfy all of the following:%
\begin{equation*}
a = qb + r, q \geq 0, 0 \leq r \lt b\text{.}
\end{equation*}
%
\end{theorem}
\begin{inlineexercise}{}{x:exercise:exe-review-division-algorithm}%
Find \(q\) and \(r\) that satisfy the Division Algorithm for the following pairs of numbers \(a\) and \(b\):%
\begin{enumerate}[label=(\alph*)]
\item{}\(\displaystyle a = 140, b = 22\)%
\item{}\(\displaystyle a = 22, b = 140\)%
\item{}\(\displaystyle a = 735, b = 21\)%
\end{enumerate}
%
\end{inlineexercise}
\begin{definition}{GCD.}{x:definition:def-gcd}%
Given integers \(a\) and \(b\) not both zero, their \terminology{greatest common divisor}, denoted by%
\begin{equation*}
\gcd(a,b)\text{,}
\end{equation*}
is the largest integer that divides both numbers.%
\par
We say that \(a\) and \(b\) are \terminology{relatively prime} if \(\gcd(a,b) = 1\).%
\end{definition}
There are a number of ways to determine the GCD of two numbers \(a\) and \(b\):%
\begin{itemize}[label=\textbullet]
\item{}Listing all factors of \(a\) and \(b\), then finding the largest one they have in common;%
\item{}Writing out the prime factorizations of \(a\) and \(b\), then collecting all common prime factors;%
\item{}The Euclidean Algorithm (repeated division).%
\end{itemize}
%
\begin{inlineexercise}{}{x:exercise:exe-review-gcd}%
Apply the three techniques above to compute \(\gcd(220,360)\).%
\par\smallskip%
\noindent\textbf{\blocktitlefont Answer}.\hypertarget{g:answer:id238017}{}\quad{}The GCD is 20.%
\par\smallskip%
\noindent\textbf{\blocktitlefont Solution}.\hypertarget{g:solution:id238070}{}\quad{}The Euclidean Algorithm performs the following steps:%
\begin{align*}
360 \amp = 1 \cdot 220 + 140 \\
220 \amp = 1 \cdot 140 + 80 \\
140 \amp = 1 \cdot 80 + 60 \\
80 \amp = 1 \cdot 60 + 20 \\
60 \amp = 3 \cdot 20 + 0 
\end{align*}
The GCD is the divisor in the last line (where the remainder becomes zero).%
\end{inlineexercise}
\begin{theorem}{Bezout's Identity.}{}{x:theorem:thm-bezout}%
Let \(a,b \in \mathbb{Z}\), not both zero. Then there exist \(m,n \in \mathbb{Z}\) such that \(am + bn = \gcd(a,b)\).%
\end{theorem}
\begin{inlineexercise}{}{x:exercise:exe-review-bezout}%
Find a pair of integers \(x\) and \(y\) such that%
\begin{equation*}
13x + 11y = 2\text{.}
\end{equation*}
Then explain why the equation \(13x + 11y = 2\) has infinitely many solutions. Can you characterize all such solutions?%
\end{inlineexercise}
\begin{inlineexercise}{}{x:exercise:exe-review-bezout-no-solution}%
Prove that the equation%
\begin{equation*}
14x - 35y = 9
\end{equation*}
has no integer solutions.%
\end{inlineexercise}
\begin{inlineexercise}{}{x:exercise:exe-review-bezout-gcd}%
Let \(a, b, d, \in \mathbb{N}\). Prove that \(ax + by = d\) has integer solutions \(x\) and \(y\) if and only if \(\gcd(a,b) \mid d\).%
\end{inlineexercise}
\begin{lemma}{Euclid's Lemma.}{}{x:lemma:lem-euclid}%
If \(p\) is prime, and \(a\) and \(b\) are integers such that \(p \mid ab\), then either \(p \mid a\) or \(p \mid b\) (or both).%
\end{lemma}
\begin{inlineexercise}{}{x:exercise:exe-review-euclid-proof}%
Prove \hyperref[x:lemma:lem-euclid]{Lemma~{\xreffont\ref{x:lemma:lem-euclid}}} using \hyperref[x:theorem:thm-bezout]{Theorem~{\xreffont\ref{x:theorem:thm-bezout}}}.%
\end{inlineexercise}
\begin{inlineexercise}{}{x:exercise:exe-review-bezout-euclid}%
Let \(m,a,b \in \mathbb{N}\). Using \hyperref[x:theorem:thm-bezout]{Theorem~{\xreffont\ref{x:theorem:thm-bezout}}}, prove that if \(m \mid ab\) and \(\gcd(a,m) = 1\), then \(m \mid b\).%
\end{inlineexercise}
\end{sectionptx}
%
%
\typeout{************************************************}
\typeout{Section 1.4 Reading and Writing Proofs}
\typeout{************************************************}
%
\begin{sectionptx}{Reading and Writing Proofs}{}{Reading and Writing Proofs}{}{}{x:section:sec-reading-writing-proofs}
Proofs are a form of communication. They are used to argue (that a claim holds true); and sometimes explain (\emph{why} a claim is true).%
\begin{activity}{}{g:activity:id238452}%
When writing proofs, one needs to be \emph{clear}, \emph{complete}, and \emph{correct}. Comment on the three proofs below of the statement%
\begin{equation*}
\text{Let } m \in \mathbb{Z}\text{. Then }m\text{ is even if and only if }m^3\text{is even}\text{.}
\end{equation*}
Which one(s) is\slash{}are%
\begin{multicols}{2}
\begin{itemize}[label=\textbullet]
\item{}convincing?%
\item{}missing steps?%
\item{}easiest to read?%
\item{}correct?%
\end{itemize}
\end{multicols}
%
\par
\alert{Proof 1} (\(\Rightarrow\)) \(m = 2k \Rightarrow m^3 = 8k^3 = 2(4k^3)\), so \(m^3\) is even.%
\par
(\(\Leftarrow\)) By contradiction. Assume \(m^3\) is even but \(m\) is odd. Then \(m = 2k + 1 \Rightarrow m^3 =(2k+1)^3 = 8k^3 + 12k^2 + 6k + 1\), which is odd.%
\par
Contradiction, so \(m\) must be even. So \(m\) even \(\Leftrightarrow\) \(m^3\) even.%
\par
\alert{Proof 2}%
\begin{equation*}
m \text{ even } \Leftrightarrow m = 2k \Leftrightarrow m^3 = 8k^3 \Leftrightarrow  m^3 = 2(4k^3) \Leftrightarrow  m^3 \text{ even}\text{,}
\end{equation*}
hence proven.%
\par
\alert{Proof 3} We want to prove that%
\begin{equation*}
m \text{ is even if and only if } m^3 \text{ is even}\text{.}
\end{equation*}
%
\par
(\(\Rightarrow\)) First, we show \(m\) even \(\Rightarrow\) \(m^3\) even. If \(m\) is even, then we can write \(m = 2k\) for some \(k \in \mathbb{Z}\). Then,%
\begin{align*}
m^3 \amp = (2k)^3\\
\amp = 8k^3\\
\amp = 2(4k^3)\text{.}
\end{align*}
Since \(k \in \mathbb{Z}\), then \(4k^3 \in \mathbb{Z}\) as well, and so \(m^3\) is even.%
\par
(\(\Leftarrow\)) Next, we prove that \(m^3\) even \(\Rightarrow\) \(m\) even. To do this we prove the contrapositive: If \(m\) is odd, then \(m^3\) is odd.%
\par
If \(m\) is odd, then there is a \(k \in \mathbb{Z}\) such that \(m = 2k + 1\). Then%
\begin{align*}
m^3 \amp = (2k+1)^3\\
\amp = 8k^3 + 12k^2 + 6k + 1\\
\amp = 2(4k^3 + 6k^2 + 3k) + 1\text{.}
\end{align*}
Since \(k \in \mathbb{Z}\), then \(4k^3 + 6k^2 + 3k \in \mathbb{Z}\) as well. Therefore \(m^3\) is odd. This completes the proof.%
\end{activity}
The use of English words makes proofs more approachable and understandable. Here are some commonly-used phrases in mathematical proofs. Note the use of the plural \emph{we} instead of the singular \emph{I}.%
\begin{sidebyside}{2}{0}{0}{0}%
\begin{sbspanel}{0.5}[center]%
\alert{Declare intentions}%
\begin{itemize}[label=\textbullet]
\item{}We will prove...%
\item{}We want to show that...%
\item{}In order to prove... we...%
\item{}At this point we need to find...%
\item{}We consider the following cases...%
\end{itemize}
%
\end{sbspanel}%
\begin{sbspanel}{0.5}[center]%
\par
\alert{Clarify implications}%
\begin{itemize}[label=\textbullet]
\item{}Since... then...%
\item{}Because..., we have...%
\item{}Therefore\slash{}thus\slash{}hence...%
\item{}This means that...%
\item{}The previous statement implies...%
\end{itemize}
%
\end{sbspanel}%
\end{sidebyside}%
\par
\alert{Explain steps}%
\begin{itemize}[label=\textbullet]
\item{}By assumption, we know that...%
\item{}By simplication\slash{}manipulation,rearranging,...%
\item{}Because of property\slash{}theorem\slash{}definition, we have...%
\end{itemize}
%
\begin{activity}{}{x:activity:exe-review-proof-writing}%
Write a complete and convincing proof of the following claim that uses the mathematical statements given below (in some order). Note that this is a proof by contradiction.%
\par
\alert{Claim}: Let \(A, B\) be subsets of some universal set \(U\). Prove that if \((A \cup B)^c = A^c \cup B^c\), then \(A \subseteq B\).%
\par
%
\begin{align*}
\amp x \in A \amp \amp x \in A^c \cup B^c\\
\amp x \not\in B \amp \amp x \in A \cup B\\
\amp x \in B^c \amp \amp x \not\in (A \cup B)^c
\end{align*}
%
\end{activity}
\end{sectionptx}
\end{chapterptx}
%
%
\typeout{************************************************}
\typeout{Chapter 2 Counting Techniques}
\typeout{************************************************}
%
\begin{chapterptx}{Counting Techniques}{}{Counting Techniques}{}{}{x:chapter:chap-counting}
\begin{objectives}{Objectives}{g:objectives:id239187}
%
\begin{itemize}[label=\textbullet]
\item{}State the Sum Rule and the Product Rule, and use them to solve counting problems.%
\item{}Derive the formulas for permutations and combinations (of \(n\) objects taken \(k\) at a time), multi-permutations, and permutations and combinations with repetition allowed.%
\item{}Given a word problem, recognize which of the above techniques is applicable, and use it to solve the problem.%
\item{}Prove simple combinatorial identities.%
\end{itemize}
\end{objectives}
%
%
\typeout{************************************************}
\typeout{Section 2.1 The Basic Counting Principles}
\typeout{************************************************}
%
\begin{sectionptx}{The Basic Counting Principles}{}{The Basic Counting Principles}{}{}{x:section:sec-basic-counting}
This is a longer sentence that is followed by another sentence. Two sentences, and a second paragraph to follow.%
\par
Let's end with some mathematics.%
\par
If the two sides of a right triangle have lengths \(a\) and \(b\) and the hypotenuse has length \(c\), then the equation%
\begin{equation*}
a^2 + b^2 = c^2
\end{equation*}
will always hold.%
\end{sectionptx}
\end{chapterptx}
\end{document}